\section{Module de croissance du zooplancton}

\subsection{Description du module}
\par{
Dans les travaux pratiques précédents on a analysé la croissance du phytoplancton en fonction des
nutriments disponibles. Dans les modèles actuels nous sommes moins intéressés par l'approvisionnement
alimentaire du phytoplancton. Nous avons plutôt procédé par enquêter l'interaction du phytoplancton avec
un prédateur naturel de cette espèce, le zooplancton. En conséquence le modèle mathèmatique a été simplifié
de manière à ce qu'on considère maintenant une offre des nutriments infiniment grand pour le
phytoplancton\footnote{En d'autres termes, la croissance du phytoplancton n'est plus limitée par les
nutriments}.
\par{
Pour mieux interpréter les simulations du modèle mathématique nous avons émis l'hypothèse que la mortalité
du phytoplancton est uniquement due au broutage du zooplancton. La mortalité du phytoplancton par d'autres
prédateurs, toxines environnementales, lyse cellulaire etc. n'est pas représentée par le modèle.
Nous supposons donc que l'influence de ces causes de mortalité peut être négligée par rapport à l'influence
de la mortalité de phytoplancton due au broutage du zooplancton.
}
\par{
En résumé, nous obtenons le modèle conceptuel effectué dans la figure~\ref{fig:partie1DiagConcept}
et les équations suivantes:
}

\begin{equation}
  {{d[DA]}\over{dt}} =
  \mu_{DA} [DA] - graz_{MSZ} [MSZ]
  \label{eq:partie1DiffEq1}
\end{equation}
\begin{equation}
  {{d[MSZ]}\over{dt}} =
  \left (
    (1- eges_{MSZ}) graz_{MSZ} Y_{MSZ} - mm_{MSZ}
  \right ) [MSZ]
  \label{eq:partie1DiffEq2}
\end{equation}

\par{
Dans la première partie des travaux pratiques, l'intérêt principal est de comparer l'impacte de la
fonctions de broutage. Pour les analyses, les fonctions de broutage suivantes ont été considérées:
}

\begin{equation}
  graz_{MSZ} = g_{MSZ} \max(T) {{[DA]}\over{kg_{MSZ}+[DA]}}
  \label{eq:partie1GrazMic}
\end{equation}
\begin{equation}
  graz_{MSZ} = g_{MSZ} \max(T) {{[DA]-[DA_0]}\over{kg_{MSZ}+([DA]-[DA_0])}}
  \label{eq:partie1GrazMicSeul}
\end{equation}
\begin{equation}
  graz_{MSZ} = g_{MSZ} \max(T) {{[DA]^2}\over{kg_{MSZ}^2+[DA]^2}}
  \label{eq:partie1GrazHol}
\end{equation}

\begin{figure}[h!]
  \includegraphics[width=\textwidth]{partie1/diagrammeConceptuel.png}
  \caption{Le modèle conceptuel du système étudié dans le quatrième cours. La croissance du phytoplancton
n'est plus limitée par les nutriments disponibles. Cependant, la croissance est toujours limitée par la
disponibilité de la lumière et de la température. Dans les formules, l'influence de ces deux impacts
environnementals est cachée dans le terme $\mu_{DA}$. Une partie du phytoplancton existant est consommer
par le mésozooplancton (=MSZ). Cette partie est représentée par les fonctions de broutage
différentes fois la concentration du mésozooplancton ($graz_{MSZ}[MSZ]$). Dans le système modèlise
le mésozooplancton n'a pas des prédateurs naturels. Donc, la mort cellulaire naturelle du zooplancton ne peut
plus être négligée (dans les formules c'est le terme $mm_{MSZ}$ qui décrit cette l'influence).
}
  \label{fig:partie1DiagConcept}
\end{figure}

\par{
Pour peut mieux comprendre les équations du modèle, la grille~\ref{tab:partie1signifParam} peut également
être intéressante. Elle donne un aperçu de la signification etc. de chaque terme dans les équations.
}
\par{
Nous voulons encore pointer que les $T_{opt}$ et $d_{opt}$ ont les mêmes valeurs pour les diatomées et le
mésozooplancton simulé. On a donc considerer que les deux espèces sont limitées par la tempèrature
de la même façon.
}

\begin{table}[h!]
\begin{center}
\begin{tabular}{ | c | c | c | c | c | }
\hline
Terme & Signification & Type & Valeur & Unité \\
\hline
$[DA]$ & \pbox{4cm}{Concentration du carbon des diatomées} & Variable d'état & $5$ & ${{mmol C}\over{m^{-3}}}$ \\
$[MSZ]$ & \pbox{4cm}{Concentration du carbon du mésozooplancton}  & Variable d'état & $1$ & ${{mmol C}\over{m^{-3}}}$ \\
$\mu_{DA}$ & \pbox{4cm}{Taux de croissance des diatomées} & \pbox{3cm}{Fonction $\mu_{max}max(T)llum$} & \pbox{4cm}{Dépend de la disponibilité de la lumière (en fonction du $[DA]$) et de la tempèrature} & $Jour^{-1}$ \\
$graz_{MSZ}$ & \pbox{4cm}{Fonction de grazing} & Fonction & \pbox{4cm}{Dépend de la tempèrature et de $[DA]$} & $Jour^{-1}$ \\
$eges_{MSZ}$ & \pbox{4cm}{Taux d'egestion du mésozooplancton} & Paramètre & $0.1$ & $-$ \\
$Y_{MSZ}$ & \pbox{4cm}{Efficience de croissance du mésozooplancton} & Paramètre & $0.25$ & $-$ \\
$mm_{MSZ}$ & \pbox{4cm}{Taux de mortalité du mésozooplancton} & Paramètre & $0.05$ & $Jour^{-1}$ \\
$g_{MSZ}$ & \pbox{4cm}{Taux de grazing maximal} & Paramètre & $1.2$ & $Jour^{-1}$ \\
$max(T)$ & \pbox{4cm}{Fonction de la limitation de la tempèrature} & \pbox{3cm}{Fonction\\(bell-shaped)} & \pbox{4cm}{Dépend de $T, T_{opt}$ et $d_{opt}$} & $-$ \\
$kg_{MSZ}$ & \pbox{4cm}{Constante de grazing} & Paramètre & $10$ & ${{mmol C}\over{m^{-3}}}$ \\
$[DA_0]$ & \pbox{4cm}{Concentration minimale avant le mésozooplancton commencent de consummer les diatomées} & Paramètre & $5$ & ${{mmol C}\over{m^{-3}}}$ \\
$\mu_{max}$ & \pbox{4cm}{Taux de croissance maximal des diatomées} & Paramètre & $1.2$ & $Jour^{-1}$ \\
$T_{opt}$ & \pbox{4cm}{Température optimale (pour les diatomées et du mésozooplancton)} & Paramètre & $16.3$ & $^{\circ}C$ \\
$d_{opt}$ & \pbox{4cm}{Delta T (pour les diatomées et du mésozooplancton)} & Paramètre & $13.7$ & $^{\circ}C$ \\
$T$ & \pbox{4cm}{Température simulée} & Paramètre & $10.0$ & $^{\circ}C$ \\
$llum$ & \pbox{4cm}{Limitation par la lumière} & \pbox{3cm}{Fonction\\$1-e^{-\alpha PAR_Z / \mu_{max}}$} & \pbox{4cm}{Dépend de la lumière disponible, ...} & $-$ \\
\end{tabular}
\end{center}
\end{table}
\clearpage
\begin{table}[h!]
\begin{center}
\begin{tabular}{ | c | c | c | c | c | }
$\alpha$ & \pbox{4cm}{L'efficacité des chloroplastes des diatomées} & Paramètre & $0.02$ & ${mol^{-1}*m^2*sec}\over{quanta * jour}$ \\
$PAR_Z$ & \pbox{4cm}{Lumière disponible par mol chlorophylle} & \pbox{3cm}{Fonction\\$PAR_0*e^{-k_e*z}$} & \pbox{4cm}{Dépend de la latitude, les solides en suspension, ...} & ${mol*quanta}\over{m^2*sec}$ \\
$PAR_0$ & \pbox{4cm}{Lumière solaire incidente} & Paramètre & $23.0$ & ${quanta}\over{m^2*sec}$ \\
$k_e$ & \pbox{4cm}{Coefficient d'extinction verticale de la lumière} & \pbox{3cm}{Fonction\\$0.35+0.02{{1}\over{2}}[DA]$} & \pbox{4cm}{Dépend des solides en suspension, ...} & $^{mol}/_m$ \\
$z$ & \pbox{4cm}{Profondeur de l'habitat} & Paramètre & $1.0$ & $m$ \\
\hline
\end{tabular}
\end{center}
  \caption{Signification, type, valeur et unité de chaque terme/paramètre dans les
équations~\ref{eq:partie1DiffEq1},~\ref{eq:partie1DiffEq2},~\ref{eq:partie1GrazMic},~\ref{eq:partie1GrazMicSeul} et~\ref{eq:partie1GrazHol}.}
  \label{tab:partie1signifParam}
\end{table}
\FloatBarrier

\subsection{Analyse mathèmatique}

\par{
La grille~\ref{tab:partie1etatsStat} donne un aperçu des états stationnaires du système
pour les fonctions de broutage differents. Les constantes supplémentaires utilisées
dans la grille sont définies comme suit:
}
\[
m = mm_{MSZ_{MAX_0}} e ^{- \left ( {{t-t_{opt}}\over{d_{opt}}} \right )}
\]
\[
k = kg_{MSZ}
\]
\[
c = \left ( 1 - eges_{MSZ} \right ) y_{MSZ}
g_{MSZ_{MAX_0}} e ^{- \left ( {{t-t_{opt}}\over{d_{opt}}} \right )}
\]
\[
f = g_{MSZ_{MAX_0}} e ^{- \left ( {{t-t_{opt}}\over{d_{opt}}} \right )}
\]

\begin{table}[h!]
\begin{center}
\begin{tabular}{ | c | c c | }
\hline
fct. de broutage & $[DA]$ & $[MSZ]$ \\
\hline
MIC, MIC\_Seul, HOL & 0 & 0 \\
MIC & $k * {{m}\over{c-m}}$ & ${{\mu_{DA} k \left ( 1 + {{m}\over{c-m}} \right )}\over{f}}$ \\
MIC\_Seuil & ${{m/c*k-m/c[DA_0]+[DA_0]}\over{1-m/c}}$ & ${\mu_{DA} * [DA] * (k + [DA] - [DA_0])}\over{f *([DA] - [DA_0])}$ \\
HOL & $k*\sqrt{{{m}\over{c-m}}}$ & ${{\mu_{DA}k \left ( 1 + {m}\over{c-m} \right )}\over{f \sqrt{{m}\over{c-m}}}}$ \\
\hline
\end{tabular}
\end{center}
  \caption{Les états stationnaires du système pour les fonctions de broutage differents. Les
constantes supplémentaires utilisées ont était définies dans le texte précédent. Les formules sont en accord
avec les simulations informatiques décrites ci-dessous.
}
  \label{tab:partie1etatsStat}
\end{table}
\FloatBarrier

\subsection{Simulation de référence}

\begin{figure}[h!]
  \includegraphics[width=\textwidth]{partie1/grazingFct.png}
  \caption{
La courbe de les trois fonctions~\ref{eq:partie1GrazMic} (MIC),~\ref{eq:partie1GrazMicSeul} (MIC\_SEUIL)
et~\ref{eq:partie1GrazHol} (HOL) en fonction de la concentration $[DA]$.
}
  \label{fig:partie1grazingFcts}
\end{figure}

\begin{figure}[h!]
  \includegraphics[width=0.5\textwidth]{partie1/refMic35.png}\hfill
  \includegraphics[width=0.5\textwidth]{partie1/refMicSeul35.png}\\
  \includegraphics[width=0.5\textwidth]{partie1/refHol35.png}\hfill
  \includegraphics[width=0.5\textwidth]{partie1/refMic150.png}\\
  \includegraphics[width=0.5\textwidth]{partie1/refMicSeul150.png}\hfill
  \includegraphics[width=0.5\textwidth]{partie1/refHol150.png}
  \caption{\todo}
  \label{fig:partie1RefSimulations}
\end{figure}

\subsection{Simulation de test 1}

